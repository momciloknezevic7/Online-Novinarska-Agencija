\documentclass{article}
\usepackage[utf8]{inputenc}
\usepackage[english]{babel}
\usepackage{indentfirst}
\usepackage{verbatim}

\usepackage{xcolor}
\usepackage{graphicx}
\renewcommand{\figurename}{Slika}
\graphicspath{ {./Slike/} }

\title{INFORMACIONI SISTEMI\\Online Novinarska Agencija}
\author{
Boris Cvitak 1022/21\\
David Nestorović 1083/22\\
Momčilo Knežević 1087/22
}

\date{Beograd, 2022.}

\begin{document}

\maketitle

\newpage

\tableofcontents

\newpage

\section{Uvod}
\indent Rad predstavlja projekat iz predmeta Informacioni sistemi na master studijama Matematičkog 
fakulteta. Rad opisuje informacioni sistem online novinarske agencije. \\
\indent Obzirom da živimo u vremenu gde su informacije ključne za svakodnevno funkcionisanje, ovakav sistem omogućuje da se informacije pravovremeno i efikasno šire tako da budu dostupne što većem broju ljudi.  Ideja je, da se u eri pametnih telefona, uspostavi direktna veza između novinarskih agencija i korisnika.

\section{Analiza sistema}
Informacioni sistem Online novinarske agencije omogućava pravovremeno širenje informacija iz raznih sfera interesovanja, putem interneta. Tok informacija kretao bi se od novinarskih agencija ka čitaocima. Pored toga, bitan aspekt sistema bio bi i komunikacija novinara (dopisnika) sa novinarskom agencijom, kao i procesuiranje dobijenih vesti od strane urednika agencije. Takođe bilo bi poželjno da sistem omogućava neki vid povratne informacije od korisnika ka samoj agenciji (komentari, ocene članaka...)  \\
\indent Činioci sistema u opštem slučaju su: \\
\indent \textbf{Novinar} - čija je uloga da prikuplja i šalje vesti, o nekom događaju, uredniku agencije u kojoj je zaposlen. Svaka vest, trebala bi da odgovori na šest novinarskih pitanja: Ko? Šta? Kada? Gde? Zašto? Kako? Pored toga, vest može sadržati i fotografiju i/ili video snimak koji bi dodatno privukli pažnju korisnika. \\
\indent \textbf{Administrator} - ima ulogu da dobijene vesti skladišti u bazi, i pruži mogućnost da se vesti mogu pretraživati ili obrisati na zahtev urednika. Takođe administrator bi trebao da omogući novinarima interfjes ka bazi kako bi mogli da pretražuju objavljene vesti i teme na kojima rade drugi novinari (da se vesti ne bi ponavljale) \\
\indent \textbf{Urednik} - čija je uloga da procesuira dobijene vesti, donese odluku koje od njih će bit objavljene, i vrši korekcije ukoliko je to potrebno. Pored toga, on može obavljati i funckiju novinara, odnosno može i sam pisati vesti. U pogledu interakcije sa čitaocima, uloga urednika je i da vrši recenziranje pristiglih komentara korisnika. \\
\indent \textbf{Čitalac} - koji koristi sistem za čitanje objavljenih vesti. Pored toga, registrovani čitaoci mogu postavljati komentare, davati ocene za objavljene vesti, i imati uvid u komentare drugih čitalaca.
\indent Na Slici 1 se nalazi dijagram koji prikazuje učesnike sistema i njihove akcije.

\newpage

\begin{figure}[htbp!]
\centering
\includegraphics[scale=0.7]{Slucajevi upotrebe}
\caption{Slučajevi upotrebe}
\label{slk:dtp}
\end{figure}

\newpage

\section{Slučajevi upotrebe}
\indent Posmatrani sistem identifikuje sledeće slučajeve upotrebe:
\begin{itemize} 
\item Registracija
\item Autorizacija 
\item Slanje vesti
\item Pretraživanje vesti i tema
\item Manipulacija vestima u bazi
\item Selekcija vesti 
\item Manipulacija komentara
\item Čitanje vesti
\item Komentarisanje i ocenjivanje vesti

\end{itemize}

\subsection{Registracija}
\begin{itemize}
    \item \textbf{Kratak opis:} Neregistrovani korisnici sistema koji žele da se prijave na sistem treba da popune formular za prijavu, pri čemu se formular razlikuje od uloge korisnika sistema. Prikupljene informacije sistem validira i obaveštava korisnika o ishodu podnete registracije.
    \item \textbf{Učesnici:} Svi neregistrovani korisnici sistema.
    \item \textbf{Preduslovi:} Sistem je aktivan i dostupan(svi korisnici sa internet konekcijom mogu da pristupe sistemu).
    \item \textbf{Postuslovi:} Po uspešnoj registraciji, korisnici dobijaju svoje kredencijale kojima se mogu prijaviti na sistem. Svaki korisnik na osnovu svoje uloge dobija određene privilegije nad sistemom. Prikupljeni podaci čuvaju se u bazi podataka.
    \item \textbf{Osnovni tok:}
        \begin{enumerate}
            \item Korisnik otvara formular za registraciju na web stranici.
            \item Korisnik bira ulogu pod kojom želi da se registruje.
            \item Sistem na osnovu izabrane uloge generiše formular za registrovanje.
            \item Korisnik unosi tražene podatke.
            \item Sistem validira unete podatke.
            \item Sistem šalje mail za potvrdu registracije.
            \item Sistem obaveštava korisnika da je mail za potvrdu poslat.
            \item Korisnik potvrdjuje ispravnost podataka.
            \item Sistem čuva podatke u bazi podataka.
            \item Sistem obaveštava korisnika o uspešno kreiranom nalogu.
	    \end{enumerate}
    \item \textbf{Alternativni tokovi:}
        \begin{itemize}
            \item[A1.] \textbf{Neuspešna validacija podataka.} Ukoliko sistem prilikom validacije naidje na neke neispravne podatke(pogrešan format, nemogućnost utvrđivanje identiteta zaposlenog...) obaveštava korisnika o pogrešno unetim podacima i zahteva njihovu korekciju.
            \item[A2.] \textbf{Zauzeto korisničko ime.} Sistem utvrđuje dostupnost unetog korisničkog imena. U slučaju da ime nije dostupno, sistem zahteva novo ime, uz odredjene sugestije na osnovu prvog unosa.
            \item[A3.] \textbf{E-mail za potvrdu nije stigao.} Ukoliko korisnik nije dobio e-mail za potvrdu, može zahtevati od sistema ponovno slanje e-maila.
        \end{itemize}
    \item \textbf{Specijalni zahtevi:}
        \begin{itemize}
			\item Sistem zahteva odredjen kvalitet unete lozinke radi veće sigurnosti.
		\end{itemize}
	\item \textbf{Dodatne informacije:}
        \begin{itemize}
            \item  Podaci potrebni za prijavu radnika:
                \begin{itemize}
                    \item Odabir uloge(novinar, urednik, administrator)
                    \item ime
                    \item prezime
                    \item korisničko ime
                    \item broj radne knjižice
                    \item e-mail adrese
                    \item broj telefona
                    \item lozinka
                \end{itemize}
             \item  Podaci potrebni za prijavu čitaoca:
                \begin{itemize}
                    \item ime
                    \item prezime
                    \item korisničko ime
                    \item e-mail adrese
                    \item lozinka
                \end{itemize}
        \end{itemize}
\end{itemize}

\newpage

\begin{figure}[htbp!]
\centering
\includegraphics[scale=0.6]{Registracija_korisnika.jpg}
\caption{Registracija korisnika}
\label{slk:dtp}
\end{figure}

\newpage




\end{document}
